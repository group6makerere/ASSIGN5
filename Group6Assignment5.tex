\documentclass{article}
\usepackage{amsmath}
\usepackage{graphicx}
\begin{document}

	\begin{center}
		{\huge\textbf{MAKERERE UNIVERSITY}}\\
		{\large\textbf{COLLEGE OF COMPUTING AND INFORMATION SCIENCES}}\\
		\textbf{SCHOOL OF COMPUTING AND INFORMATICS TECHNOLOGY}\\
		\textbf{DEPARTMENT OF COMPUTER SCIENCE}
	\end{center}
	GROUP COURSE WORK\\
	COURSE UNIT : RESEARCH METHODOLOGY\\
	INSTRUCTOR : EARNEST MWEBAZE\\
	COURSE CODE : BIT 2207\\



\begin{tabular}{|p{5cm} |p{3cm}|p{3cm}|p{3cm}|}
\hline
\multicolumn{4}{|c|}{Group 6}\\
\hline
\hline
Name & Registration Number & Student Number&Signature\\
\hline
MATOVU GEOFREY&16/U/6902/EVE&216008997&\\
BAMANYE&16/U/4206/EVE&216013268&\\
KAJJOBA MUDASILU&16/U/5198/EVE&216015944&\\
SADALA BADRU ATULINDA&16/U/3901/EVE&216010207\\
\hline

\end{tabular}
\section{HARD DISK FAILURE DETECTOR}
\subsection{Introduction}
The aim of this document is to provide a high light of how hard drive failure prediction can be improved and summarize a range of literature to highlight themes and generate recommendations for future work in this area by Hard Drive Failure Predictor. This section contains the background which purposely describes the origin of hard disk failure and points out past and recent studies, problem statement pointing out research gaps, objectives and scope of the study. It will help clarify to the reader how fulfillment of the research aims and objectives will improve on data storage through effective hard drive failure detection. In addition the document will partitioned into other sections which include the literature review, research methodology timeframe, risks and project budget inclusive.  





\end{document}



